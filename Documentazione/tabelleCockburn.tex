Le tabelle di Cockburn fanno riferimento alla descrizione dei casi d’uso secondo Alistair Cockburn.
Sono template testuali strutturati per descrivere come un attore interagisce con il sistema per raggiungere 
un obiettivo. In pratica, ogni caso d’uso Cockburn descrive:
\begin{enumerate}
    \item chi compie l’azione (attore)
    \item cosa vuole ottenere (obiettivo)
    \item come il sistema e l’utente interagiscono passo dopo passo
    \item cosa può andare storto (flussi alternativi ed eccezioni)
\end{enumerate}
Di seguito presentiamo i diagrammi di cockburn per i seguenti requisiti funzionali
\begin{enumerate}
    \item Registrazione utente
    \item Creazione immobile
    \item Ricerca immobile
    \item Avvio della trattativa
\end{enumerate}
\footnote{Nel corso di questa documentazione, questi quattro requisiti funzionali verranno analizzati 
a diversi livelli di astrazione e granularità}




\includepdf[pages=1, pagecommand={\thispagestyle{plain}}, fitpaper=false, width=\paperwidth, height=\paperheight]{images/Ricerca Immobile.pdf}
\includepdf[pages=1, pagecommand={\thispagestyle{plain}}, fitpaper=false, width=\paperwidth, height=\paperheight]{images/Use case Registrazione Utente.pdf}
\includepdf[pages=1, pagecommand={\thispagestyle{plain}}, fitpaper=false, width=\paperwidth, height=\paperheight]{images/Inserimento di un immobile.pdf}
\includepdf[pages=1, pagecommand={\thispagestyle{plain}}, fitpaper=false, width=\paperwidth, height=\paperheight]{images/AvvioChat.pdf}
