\textbf{Requisito 1: Gestione Utenti e Autenticazione}
\textbf{Attori coinvolti}
\begin{itemize}
    \item Amministratore / gestore agenzie immobiliare
    \item Agente immobiliare
    \item Utente
\end{itemize}
\textbf{Descrizione:} \\
Il sistema deve permettere la registrazione di nuovi utenti (clienti) e (gestore di agenzie immobiliari)
Tutti gli attori devono essere in grado di effettuare il login con credenziali sicure. 
L’Amministratore deve poter modificare le credenziali di accesso predefinite e registrare gli account degli agenti immobiliari.
\\ \\
\textbf{Requisito 2: Inserimento di Inserzioni Immobiliari}
\textbf{Attori coinvolti}
\begin{itemize}
    \item Agente immobiliare
\end{itemize}
\textbf{Descrizione:} \\
Gli agenti immobiliari possono caricare nuove inserzioni di immobili, complete di dettagli 
quali foto, descrizione, prezzo, dimensioni, indirizzo, numero di stanze, classe energetica,
ecc. Le inserzioni devono essere classificate per tipologia: vendita o affitto.
\\ \\
\textbf{Requisito 3: Ricerca Avanzata di Immobili}
\textbf{Attori coinvolti}
\begin{itemize}
    \item Utente
    \item Geopify
\end{itemize}
\textbf{Descrizione:} \\
Il sistema deve permettere la ricerca avanzata di immobili tramite filtri multipli: tipologia, 
prezzo, posizione geografica (con supporto mappa), numero di stanze, classe energetica, ecc. 
La ricerca deve essere efficiente e visualizzare i risultati anche tramite mappa interattiva.
\\ \\
\textbf{Requisito 4: Gestione delle Offerte sugli Immobili}
\textbf{Attori coinvolti}
\begin{itemize}
    \item Agente immobiliare
    \item Utente registrato
\end{itemize}
\textbf{Descrizione:} \\
Gli utenti registrati possono fare offerte per immobili specificando un prezzo. Gli agenti 
possono accettare, rifiutare o fare controproposte. Deve essere possibile visualizzare uno s
torico delle offerte sia per l’utente che per l’agente. Gli agenti possono inserire manualmente 
offerte ricevute esternamente al sistema.
\\ \\
\textbf{Requisito 5: Integrazione con Servizi Esterni (Geoapify)}
\textbf{Attori coinvolti}
\begin{itemize}
    \item Geopify
\end{itemize}
\textbf{Descrizione:} \\
All’atto della creazione di un’inserzione immobiliare, il sistema deve interrogare il servizio
esterno Geoapify per verificare la presenza di scuole, parchi o trasporto pubblico 
nelle vicinanze dell’immobile. Se presenti, verranno mostrati appositi indicatori 
(“Vicino a scuole”, ecc.).
