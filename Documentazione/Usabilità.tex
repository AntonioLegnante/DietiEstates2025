Al fine di valutare l’usabilità della piattaforma DietiEstates, è stato deciso di affiancare 
alle attività di testing funzionale un questionario di usabilità rivolto agli utenti. 
Questa scelta nasce dall’esigenza di raccogliere feedback strutturati direttamente dagli 
utilizzatori finali del sistema, così da ottenere una valutazione non solo tecnica, ma anche 
esperienziale dell’interazione con la piattaforma.

L’utilizzo di un questionario consente di misurare in modo sistematico aspetti chiave 
dell’usabilità, quali la facilità di apprendimento, l’efficienza nell’esecuzione dei compiti,
la chiarezza dell’interfaccia, la comprensibilità delle funzionalità e il livello di 
soddisfazione percepito dagli utenti. A differenza di una semplice osservazione informale, il
questionario permette di standardizzare la raccolta dei dati, rendendo confrontabili le 
risposte tra utenti diversi e tra versioni differenti del sistema.

Il testing dell’usabilità rappresenta una fase fondamentale nel processo di sviluppo software,
in quanto consente di individuare criticità che non emergono dai soli test funzionali. Un 
sistema può infatti risultare corretto dal punto di vista tecnico, ma difficile da utilizzare 
o poco intuitivo per l’utente finale. Coinvolgere gli utenti nel processo di valutazione
permette di evidenziare problemi di progettazione dell’interfaccia, ambiguità nelle 
funzionalità offerte e ostacoli nel flusso di interazione, fornendo indicazioni utili
per il miglioramento iterativo del prodotto.

\newpage

\begin{figure}[h]
    \centering
    \includegraphics[width=\textwidth]{images/1.png}
    \caption{Descrizione dell'immagine}
    \label{fig:etichetta}
\end{figure}

\begin{figure}[h]
    \centering
    \includegraphics[width=\textwidth]{images/2.png}
    \caption{Descrizione dell'immagine}
    \label{fig:etichetta}
\end{figure}

\begin{figure}[h]
    \centering
    \includegraphics[width=\textwidth]{images/3.png}
    \caption{Descrizione dell'immagine}
    \label{fig:etichetta}
\end{figure}

\begin{figure}[h]
    \centering
    \includegraphics[width=\textwidth]{images/4.png}
    \caption{Descrizione dell'immagine}
    \label{fig:etichetta}
\end{figure}

\begin{figure}[h]
    \centering
    \includegraphics[width=\textwidth]{images/5.png}
    \caption{Descrizione dell'immagine}
    \label{fig:etichetta}
\end{figure}

\begin{figure}[h]
    \centering
    \includegraphics[width=\textwidth]{images/6.png}
    \caption{Descrizione dell'immagine}
    \label{fig:etichetta}
\end{figure}

\begin{figure}[h]
    \centering
    \includegraphics[width=\textwidth]{images/7.png}
    \caption{Descrizione dell'immagine}
    \label{fig:etichetta}
\end{figure}

\begin{figure}[h]
    \centering
    \includegraphics[width=\textwidth]{images/8.png}
    \caption{Descrizione dell'immagine}
    \label{fig:etichetta}
\end{figure}

