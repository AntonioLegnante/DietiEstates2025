La suite di test del progetto DietiEstates2025 adotta un approccio multi-livello combinando 
diverse strategie di testing secondo il modello Testing Pyramid:
\begin{enumerate}
    \item Unit Testing (base): Isolamento dei componenti singoli
    \item Integration Testing (middle): Test di flussi completi
    \item Controller Testing (top): Test dell'interfaccia HTTP
\end{enumerate}

Sono stati individuati diversi scenari positivi e negativi
\begin{enumerate}
    \item Scenari positivi verificano che le operazioni funzionino correttamente con input validi, 
    coprendone il maggior numero per singolo test
    \item Scenari negativi testano la gestione degli errori, isolando ogni singolo input errato per singolo test   
\end{enumerate}

Ogni test segue rigorosamente il pattern Arrange, Act, Assert:
\begin{enumerate}
    \item Arrange: setup dei mock e dei dati di test
    \item Act: esecuzione del metodo da testare
    \item Assert: verifica dei risultati attesi
\end{enumerate}

Sono state testate le seguenti funzionalità
\begin{enumerate}
    \item Registrazione
    \item Login
    \item Ricerca immobili
    \item Caricamento immobili
    \item Avvio di una trattativa
\end{enumerate}

La maggior parte dei test segue l'approccio WECT (Weak Equivalence Class Testing) 
mantenendo gli altri valori costanti
La suite utilizza @MockBean per isolare il service layer dalle sue dipendenze (repository, MinIO service). 
Questo permette di testare la logica del servizio senza dipendere da database o servizi esterni reali.

