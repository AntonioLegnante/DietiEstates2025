\textbf{Test Registrazione Cliente valida}
\begin{table}[!h]
    \centering
    \small 
    \begin{tabular}{|p{0.5cm}|p{3cm}|p{5cm}|p{5cm}|}
    \hline
    \textbf{id} & \textbf{Precond} & \textbf{Input} & \textbf{Output aspettato} \\
    \hline
    1 & Nessuna & username=newuser, \newline email=newuser@test.com, \newline password=securePassword123, \newline ruolo=cliente & 
    result!=null, \newline result.username=newuser, \newline result.email=newuser@test.com, \newline result.ruolo=cliente \\
    \hline
    \end{tabular}
\end{table}
\begin{lstlisting}[style=react]
    @Test
    @DisplayName("Test Registrazione Cliente valida")
    void testRegistrazioneClienteSuccess() {
        // Arrange
        when(utenteRepository.existsByUsername("newuser")).thenReturn(false);
        when(utenteRepository.existsByEmail("newuser@test.com")).thenReturn(false);
        when(passwordEncoder.encode(anyString())).thenReturn("hashedPassword");
        when(utenteRepository.save(any(Utente.class))).thenReturn(utente);
        // Act
        Utente result = authService.registrazione(registrazioneRequest);
        // Assert
        assertNotNull(result);
        assertEquals("newuser", result.getUsername());
        assertEquals("newuser@test.com", result.getEmail());
        assertEquals("cliente", result.getRuolo());
        verify(utenteRepository, times(1)).save(any(Utente.class));
\end{lstlisting}

\newpage

\textbf{Test Registrazione con username duplicato}
\begin{table}[!h]
    \centering
    \small 
    \begin{tabular}{|p{0.5cm}|p{3cm}|p{5cm}|p{5cm}|}
    \hline
    \textbf{id} & \textbf{Precond} & \textbf{Input} & \textbf{Output aspettato} \\
    \hline
    2 & utente con username=newuser già presente & username=newuser, \newline email=newuser@test.com, \newline password=securePassword123, \newline ruolo=cliente & 
    lancia IllegalArgumentException \\
    \hline
    \end{tabular}
\end{table}

\begin{lstlisting}[style=react]
    @Test
    @DisplayName("Test Registrazione con username duplicato")
    void testRegistrazioneFailureDuplicateUsername() {
        // Arrange
        when(utenteRepository.existsByUsername("newuser")).thenReturn(true);
        // Act & Assert
        assertThrows(RuntimeException.class, () -> {
            authService.registrazione(registrazioneRequest);
        }, "Username gia esistente");
        verify(utenteRepository, never()).save(any());
    }
\end{lstlisting}

\textbf{Test Registrazione con email duplicata}
\begin{table}[!h]
    \centering
    \small 
    \begin{tabular}{|p{0.5cm}|p{3cm}|p{5cm}|p{5cm}|}
    \hline
    \textbf{id} & \textbf{Precond} & \textbf{Input} & \textbf{Output aspettato} \\
    \hline
    3 & utente con username diverso da newuser e email=newuser@test.- com già presente & username=newuser, \newline email=newuser@test.com, \newline password=securePassword123, \newline ruolo=cliente & 
    lancia IllegalArgumentException \\
    \hline
    \end{tabular}
\end{table}
\begin{lstlisting}[style=react]
    @Test
    @DisplayName("Test Registrazione con email duplicata")
    void testRegistrazioneFailureDuplicateEmail() {
        // Arrange
        when(utenteRepository.existsByUsername("newuser")).thenReturn(false);
        when(utenteRepository.existsByEmail("newuser@test.com")).thenReturn(true);
        // Act & Assert
        assertThrows(RuntimeException.class, () -> {
            authService.registrazione(registrazioneRequest);
        }, "Email gia registrata");
        verify(utenteRepository, never()).save(any());
    }
\end{lstlisting}

\newpage
\textbf{Test Registrazione Gestore con Agenzia valida}
\begin{table}[!h]
    \centering
    \small 
    \begin{tabular}{|p{0.5cm}|p{3cm}|p{5cm}|p{5cm}|}
    \hline
    \textbf{id} & \textbf{Precond} & \textbf{Input} & \textbf{Output aspettato} \\
    \hline
    4 & Nessuna & username=newuser, \newline email=newuser@test.com, \newline password=securePassword123, \newline ruolo=nuovoAmministratore,
    \newline nomeAgenzia=Test Agency, \newline indirizzoAgenzia=Via Test 1, \newline cittaAgenzia=Napoli, \newline telefonoAgenzia=0815551234, 
    \newline emailAgenzia=agency@test.com, \newline partitaIVA=12345678901 & 
    result!=null, \newline result.ruolo=nuovoAmministratore, \newline result.agenziaGestita.nomeAgenzia -=Test Agency \\
    \hline
    \end{tabular}
\end{table}
\begin{lstlisting}[style=react]
    @Test
    @DisplayName("Test Registrazione Gestore con Agenzia valida")
    void testRegistrazioneGestoreSuccess() {
        // Arrange
        when(utenteRepository.existsByUsername("newuser")).thenReturn(false);
        when(utenteRepository.existsByEmail("newuser@test.com")).thenReturn(false);
        when(agenziaRepository.existsByNomeAgenzia("Test Agency")).thenReturn(false);
        when(agenziaRepository.existsByPartitaIVA("12345678901")).thenReturn(false);
        when(passwordEncoder.encode(anyString())).thenReturn("hashedPassword");
        when(utenteRepository.save(any(Utente.class))).thenReturn(gestore);
        when(agenziaRepository.save(any(Agenzia.class))).thenReturn(agenzia);
        // Act
        Utente result = authService.registrazione(registrazioneRequest);
        // Assert
        assertNotNull(result);
        assertEquals("nuovoAmministratore", result.getRuolo());
        assertEquals("Test Agency", result.getAgenziaGestita().getNomeAgenzia());
    }
\end{lstlisting}
\newpage

\textbf{Test Registrazione Gestore senza dati agenzia}
\begin{table}[!h]
    \centering
    \small 
    \begin{tabular}{|p{0.5cm}|p{3cm}|p{5cm}|p{5cm}|}
    \hline
    \textbf{id} & \textbf{Precond} & \textbf{Input} & \textbf{Output aspettato} \\
    \hline
    5 & Nessuna & username=newuser, \newline email=newuser@test.com, \newline password=securePassword123, \newline ruolo=nuovoAmministratore,
    \newline agenzia=null & lancia IllegalArgumentException \\
    \hline
    \end{tabular}
\end{table}
\begin{lstlisting}[style=react]
    @Test
    @DisplayName("Test Registrazione Gestore senza dati agenzia")
    void testRegistrazioneGestoreFailureNoAgencyData() {
        // Arrange
        registrazioneRequest.setRuolo("nuovoAmministratore");
        registrazioneRequest.setAgenzia(null);
        when(utenteRepository.existsByUsername("newuser")).thenReturn(false);
        when(utenteRepository.existsByEmail("newuser@test.com")).thenReturn(false);
        // Act & Assert
        assertThrows(RuntimeException.class, () -> {
            authService.registrazione(registrazioneRequest);
        }, "Dati agenzia mancanti");
    }
\end{lstlisting}

\newpage
\textbf{Test Registrazione con nome agenzia duplicato}
\begin{table}[!h]
    \centering
    \small 
    \begin{tabular}{|p{0.5cm}|p{3cm}|p{5cm}|p{5cm}|}
    \hline
    \textbf{id} & \textbf{Precond} & \textbf{Input} & \textbf{Output aspettato} \\
    \hline
    6 & Agenzia con nome Existing Agency già presente nel sistema & username=newuser, \newline email=newuser@test.com, \newline password=securePassword123, \newline ruolo=nuovoAmministratore,
    \newline nomeAgenzia=Existing Agency \newline indirizzoAgenzia=Via Test 1, \newline cittaAgenzia=Napoli, \newline telefonoAgenzia=0815551234, 
    \newline emailAgenzia=agency@test.com, \newline partitaIVA=99999999999 & lancia RuntimeException \\
    \hline
    \end{tabular}
\end{table}

\begin{lstlisting}[style=react]
    @Test
    @DisplayName("Test Registrazione con nome agenzia duplicato")
    void testRegistrazioneGestoreFailureDuplicateAgencyName() {
        // Arrange
        when(utenteRepository.existsByUsername("newuser")).thenReturn(false);
        when(utenteRepository.existsByEmail("newuser@test.com")).thenReturn(false);
        when(agenziaRepository.existsByNomeAgenzia("Existing Agency")).thenReturn(true);
        // Act & Assert
        assertThrows(RuntimeException.class, () -> {
            authService.registrazione(registrazioneRequest);
        }, "Nome agenzia gia esistente");
    }
\end{lstlisting}
\newpage
