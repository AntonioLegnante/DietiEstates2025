\textbf{Test Ricerca Immobili per città}
\begin{table}[!h]
    \centering
    \small 
    \begin{tabular}{|p{0.5cm}|p{3cm}|p{5cm}|p{5cm}|}
    \hline
    \textbf{id} & \textbf{Precond} & \textbf{Input} & \textbf{Output aspettato} \\
    \hline
    11 & Nella piattaforma è presente un immobile localizzato a Napoli & localita=Napoli, \newline minPrezzo=null, \newline maxPrezzo=null, \newline affitta=false
    \newline vendita=true, \newline numeroStanze=null, \newline dimensione=null, \newline piano=null, \newline classeEnergetica=null, numeroBagni=null & 
    result!=null, \newline result.size=1, \newline result.get(1).citta=Napoli \\
    \hline
    \end{tabular}
\end{table}

\begin{lstlisting}[style=react]
   @Test
   @DisplayName("Test Ricerca Immobili per citta")
    void testSearchImmobiliByCity() {
        // Arrange
        List<Immobile> immobili = Arrays.asList(immobile);
        when(immobileRepository.ricercaAvanzata(
                "Napoli", null, null, false, true, null, null,
                null, null, null
        )).thenReturn(immobili);
        // Act
        List<Immobile> result = immobileService.applicaRicerca(
                "Napoli", null, null, false, true, null, null,
                null, null, null
        );
        // Assert
        assertNotNull(result);
        assertEquals(1, result.size());
        assertEquals("Napoli", result.get(0).getCitta());
    }
\end{lstlisting}

\textbf{Test Ricerca Immobili per intervallo prezzo}
\begin{table}[!h]
    \centering
    \small 
    \begin{tabular}{|p{0.5cm}|p{3cm}|p{5cm}|p{5cm}|}
    \hline
    \textbf{id} & \textbf{Precond} & \textbf{Input} & \textbf{Output aspettato} \\
    \hline
    12 & Nella piattaforma è presente un immobile il cui prezzo è compreso fra 300000.0 e 400000.0 & localita=Napoli, \newline minPrezzo=300000.0, \newline maxPrezzo= 400000.0, \newline affitta=false
    \newline vendita=true, \newline numeroStanze=null, \newline dimensione=null, \newline piano=null, \newline classeEnergetica=null, numeroBagni=null & 
    result!=null, \newline result.size=1, \newline result.get(1).prezzo>=300000.0, \newline result.get(1).prezzo<=400000.0   \\
    \hline
    \end{tabular}
\end{table}

\begin{lstlisting}[style=react]
   @Test
    @DisplayName("Test Ricerca Immobili per intervallo prezzo")
    void testSearchImmobiliByPriceRange() {
        // Arrange
        List<Immobile> immobili = Arrays.asList(immobile);
        when(immobileRepository.ricercaAvanzata(
                "Napoli", 300000.0, 400000.0, false, true, null, null,
                null, null, null
        )).thenReturn(immobili);

        // Act
        List<Immobile> result = immobileService.applicaRicerca(
                "Napoli", 300000.0, 400000.0, false, true, null, null,
                null, null, null
        );

        // Assert
        assertNotNull(result);
        assertEquals(1, result.size());
        assertTrue(result.get(0).getPrezzo() >= 300000.0);
        assertTrue(result.get(0).getPrezzo() <= 400000.0);
    }
\end{lstlisting}

\newpage
\textbf{Test Ricerca Immobili in affitto}
\begin{table}[!h]
    \centering
    \small 
    \begin{tabular}{|p{0.5cm}|p{3cm}|p{5cm}|p{5cm}|}
    \hline
    \textbf{id} & \textbf{Precond} & \textbf{Input} & \textbf{Output aspettato} \\
    \hline
    13 & Nella piattaforma è presente un immobile in affitto & localita=Roma, \newline minPrezzo=null, \newline maxPrezzo=null, \newline affitta=true
    \newline vendita=false, \newline numeroStanze=null, \newline dimensione=null, \newline piano=null, \newline classeEnergetica=null, numeroBagni=null & 
    result!=null, \newline result.affitto=true  \\
    \hline
    \end{tabular}
\end{table}

\begin{lstlisting}[style=react]
   @Test
    @DisplayName("Test Ricerca Immobili in affitto")
    void testSearchImmobiliForRent() {
        // Arrange
        List<Immobile> immobili = Arrays.asList(immobileAffitto);
        when(immobileRepository.ricercaAvanzata(
                "Roma", null, null, true, false, null, null,
                null, null, null
        )).thenReturn(immobili);

        // Act
        List<Immobile> result = immobileService.applicaRicerca(
                "Roma", null, null, true, false, null, null,
                null, null, null
        );

        // Assert
        assertNotNull(result);
        assertTrue(result.get(0).getAffitto());
        assertFalse(result.get(0).getVendita());
    }
\end{lstlisting}

\newpage

\textbf{Test Ricerca Immobili in vendita}
\begin{table}[!h]
    \centering
    \small 
    \begin{tabular}{|p{0.5cm}|p{3cm}|p{5cm}|p{5cm}|}
    \hline
    \textbf{id} & \textbf{Precond} & \textbf{Input} & \textbf{Output aspettato} \\
    \hline
    14 & Nella piattaforma è presente un immobile in vedita & localita=Napoli, \newline minPrezzo=null, \newline maxPrezzo=null, \newline affitta=false
    \newline vendita=true, \newline numeroStanze=null, \newline dimensione=null, \newline piano=null, \newline classeEnergetica=null, numeroBagni=null & 
    result!=null, \newline result.vendita=true  \\
    \hline
    \end{tabular}
\end{table}

\begin{lstlisting}[style=react]
    @Test
    @DisplayName("Test Ricerca Immobili in vendita")
    void testSearchImmobiliForSale() {
        // Arrange
        List<Immobile> immobili = Arrays.asList(immobile);
        when(immobileRepository.ricercaAvanzata(
                "Napoli", null, null, false, true, null, null,
                null, null, null
        )).thenReturn(immobili);

        // Act
        List<Immobile> result = immobileService.applicaRicerca(
                "Napoli", null, null, false, true, null, null,
                null, null, null
        );

        // Assert
        assertNotNull(result);
        assertFalse(result.get(0).getAffitto());
        assertTrue(result.get(0).getVendita());
    }
\end{lstlisting}

\newpage
\textbf{Test Ricerca Immobili con risultati vuoti}
\begin{table}[!h]
    \centering
    \small 
    \begin{tabular}{|p{0.5cm}|p{3cm}|p{5cm}|p{5cm}|}
    \hline
    \textbf{id} & \textbf{Precond} & \textbf{Input} & \textbf{Output aspettato} \\
    \hline
    15 & Nella piattaforma non è presente un immobile localizzato a Palermo in vendita & localita=Palermo, \newline minPrezzo=null, \newline maxPrezzo=null, \newline affitta=false
    \newline vendita=true, \newline numeroStanze=null, \newline dimensione=null, \newline piano=null, \newline classeEnergetica=null, numeroBagni=null & 
    result!=null, \newline result.size=0  \\
    \hline
    \end{tabular}
\end{table}

\begin{lstlisting}[style=react]
     @Test
    @DisplayName("Test Ricerca Immobili con risultati vuoti")
    void testSearchImmobiliEmptyResults() {
        // Arrange
        when(immobileRepository.ricercaAvanzata(
                "Palermo", null, null, false, true, null, null,
                null, null, null
        )).thenReturn(new ArrayList<>());

        // Act
        List<Immobile> result = immobileService.applicaRicerca(
                "Palermo", null, null, false, true, null, null,
                null, null, null
        );

        // Assert
        assertNotNull(result);
        assertEquals(0, result.size());
    }
\end{lstlisting}

\newpage
\textbf{Test Ricerca Immobili con risultati vuoti}
\begin{table}[!h]
    \centering
    \small 
    \begin{tabular}{|p{0.5cm}|p{3cm}|p{5cm}|p{5cm}|}
    \hline
    \textbf{id} & \textbf{Precond} & \textbf{Input} & \textbf{Output aspettato} \\
    \hline
    16 & Nella piattaforma è presente un immobile in vendita a Napoli con 3 stanze di 120mq con prezzo compreso fra 300000.0 e 400000.0 & 
    localita=Napoli, \newline minPrezzo=300000.0, \newline maxPrezzo=400000.0, \newline affitta=false
    \newline vendita=true, \newline numeroStanze=3, \newline dimensione=120mq, \newline piano=null, \newline classeEnergetica=null, numeroBagni=null & 
    result!=null, \newline result.size=1, \newline result.get(1).citta=Napoli, \newline result.get(1).prezzo=350000.0, \newline result.get(1).numeroStanze=3,
    \newline result.get(1).dimensione=120mq  \\
    \hline
    \end{tabular}
\end{table}

\begin{lstlisting}[style=react]
    @Test
    @DisplayName("Test Ricerca con piu filtri combinati")
    void testSearchImmobiliWithMultipleFilters() {
        // Arrange
        List<Immobile> immobili = Arrays.asList(immobile);
        when(immobileRepository.ricercaAvanzata(
                "Napoli", 300000.0, 400000.0, false, true, 3, "120 mq",
                null, null, null
        )).thenReturn(immobili);
        // Act
        List<Immobile> result = immobileService.applicaRicerca(
                "Napoli", 300000.0, 400000.0, false, true, 3, "120 mq",
                null, null, null
        );
        // Assert
        assertNotNull(result);
        assertEquals(1, result.size());
        assertEquals("Napoli", result.get(0).getCitta());
        assertEquals(350000.0, result.get(0).getPrezzo());
        assertEquals(3, result.get(0).getNumeroStanze());
        assertEquals("120 mq", result.get(0).getDimensione());
    }
\end{lstlisting}





