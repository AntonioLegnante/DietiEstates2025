In questa sezione si vuole descrivere come è stata pensata l'interfaccia e perchè sono state adottate
determinate scelte piuttosto che altre.
L’interfaccia di DietiEstates25 è strutturata come un’applicazione web con una barra di navigazione 
principale e un’area centrale dedicata alla gestione di immobili
\\\\
\textbf{Navigazione}

La navigazione è organizzata tramite un menu principale che consente l’accesso diretto alle 
funzionalità principali del sistema. Quando l’utente non è autenticato, il menu è limitato alle 
pagine Home, dalla quale è possibile effettuare la ricerca degli immobili, e alle funzionalità di 
Registrazione e Accesso.

Un utente autenticato dispone di ulteriori voci di menu; in particolare, un agente immobiliare ha 
accesso alla funzionalità di inserimento di un nuovo immobile. Tutti gli utenti autenticati possono 
inoltre accedere allo storico delle chat associate ai singoli immobili tramite un’apposita voce 
del menu.
\\\\
\textbf{Motivazioni delle scelte}

L’interfaccia privilegia una disposizione semplice e lineare al fine di ridurre il carico cognitivo 
degli utenti, che non sono necessariamente esperti informatici. La pagina principale presenta una 
barra di ricerca con filtri essenziali, quali città, tipologia dell’operazione (affitto o vendita), 
prezzo minimo e prezzo massimo; ricerche più specifiche possono essere effettuate tramite l’icona 
“Altri filtri”.

L’utilizzo di maschere guidate e campi obbligatori riduce la possibilità di inserimento di dati non 
validi, come date incoerenti o importi errati, migliorando l’affidabilità complessiva del sistema.

La pagina principale, fino all’esecuzione di una ricerca, rimane intenzionalmente vuota per limitare 
la quantità di informazioni visualizzate e prevenire il disorientamento dei nuovi utenti. Ogni 
immobile è rappresentato tramite una scheda contenente le informazioni essenziali e una mappa 
dedicata; tale scelta evita l’uso di una singola mappa sovraccarica con tutti gli immobili della 
zona di riferimento, risultando di difficile consultazione. Ulteriori dettagli sono accessibili 
tramite la pagina specifica dell’immobile, raggiungibile selezionando la relativa scheda.