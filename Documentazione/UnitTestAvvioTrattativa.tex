\textbf{Inserimento di un offerta corretta - Back-end}
\begin{table}[!h]
    \centering
    \small 
    \begin{tabular}{|p{0.5cm}|p{3cm}|p{5cm}|p{5cm}|}
    \hline
    \textbf{id} & \textbf{Precond} & \textbf{Input} & \textbf{Output aspettato} \\
    \hline
    17 & Nella piattaforma è presente un immobile senza offerte & importoSpecifico=250000.0, \newline noteSpecifiche=Offerta ad hoc per appartamento vista mare & 
    listaOfferte.size=1, \newline listaOfferta.get(1).importoSpecifico= 250000.0, \newline listaOfferta.get(1).noteSpecifiche=Of -ferta ad hoc per appartamento vista mare \\
    \hline
    \end{tabular} 
\end{table}

\begin{lstlisting}[style=react]
    @Test
    @DisplayName("Inserimento di un offerta corretta")
    void testAggiungiOffertaConDatiAdHoc() {
        //Act
        chat.aggiungiOfferta(offertaMock);
        List<Offerta> listaOfferte = chat.getOfferte();
        assertEquals(1, listaOfferte.size(), "La lista dovrebbe contenere esattamente un'offerta");
        assertTrue(listaOfferte.contains(offertaMock), "L'offerta mockata dovrebbe essere nella lista della chat");
        assertEquals(importoSpecifico, listaOfferte.get(0).getImportoOfferto());
        assertEquals(noteSpecifiche, listaOfferte.get(0).getNote());
        verify(offertaMock, times(1)).setChat(chat);
    }
\end{lstlisting} 

\newpage
\textbf{Test Accettare la seconda offerta su tre}
\begin{table}[!h]
    \centering
    \small 
    \begin{tabular}{|p{0.5cm}|p{3cm}|p{5cm}|p{5cm}|}
    \hline
    \textbf{id} & \textbf{Precond} & \textbf{Input} & \textbf{Output aspettato} \\
    \hline
    17 & Nella piattaforma è presente un immobile. Su questa chat sono state caricate 3 offerte & idOfferta=2 \newline username=agente & 
    result!=null, \newline result.dataRisposta!=null, \newline offerta1.StatoOfferta=CHIUSA\_AC -CETTATA, \newline offerta3.StatoOfferta=CHIUSA\_RI -FIUTATA, 
    \newline offerta3.StatoOfferta=CHIUSA\_RI -FIUTATA \\
    \hline
    \end{tabular} 
\end{table}

\begin{lstlisting}[style=react]
    @Test
    @DisplayName("Accettare la seconda offerta su tre")
    void accettaOffertaSeconda() {
        //Arrange
        when(offertaRepository.findById(2)).thenReturn(Optional.of(o2));
        when(utenteRepository.findByUsername("agente")).thenReturn(Optional.of(vendor));
        when(offertaRepository.save(any(Offerta.class))).thenAnswer(inv -> inv.getArgument(0));
        //Act
        Offerta result = chatService.accettaOfferta(2, "agente");
        assertNotNull(result);
        assertEquals(StatoOfferta.ACCETTATA, result.getStato());
        assertNotNull(result.getDataRisposta());
        assertEquals(StatoNegoziazione.CHIUSA_ACCETTATA, chat.getStatoNegoziazione());
        assertEquals(StatoOfferta.RIFIUTATA, o1.getStato());
        assertEquals(StatoOfferta.RIFIUTATA, o3.getStato());
        verify(offertaRepository, times(1)).findById(2);
        verify(offertaRepository, times(1)).save(result);
    }
\end{lstlisting}    

\newpage
\textbf{Test Accettare la terza offerta su tre}
\begin{table}[!h]
    \centering
    \small 
    \begin{tabular}{|p{0.5cm}|p{3cm}|p{5cm}|p{5cm}|}
    \hline
    \textbf{id} & \textbf{Precond} & \textbf{Input} & \textbf{Output aspettato} \\
    \hline
    19 & Nella piattaforma è presente un immobile. Su questa chat sono state caricate 3 offerte & idOfferta=3 \newline username=agente & 
    result!=null, \newline result.dataRisposta!=null, \newline offerta1.StatoOfferta=CHIUSA\_RI -FIUTATA, \newline offerta3.StatoOfferta=CHIUSA\_AC -CETTATA, 
    \newline offerta2.StatoOfferta=CHIUSA\_RI -FIUTATA \\
    \hline
    \end{tabular} 
\end{table}

\begin{lstlisting}[style=react]
    @Test
    @DisplayName("Accettare la terza offerta su tre")
    void accettaOffertaSeconda() {
        //Arrange
        when(offertaRepository.findById(2)).thenReturn(Optional.of(o2));
        when(utenteRepository.findByUsername("agente")).thenReturn(Optional.of(vendor));
        when(offertaRepository.save(any(Offerta.class))).thenAnswer(inv -> inv.getArgument(0));
        //Act
        Offerta result = chatService.accettaOfferta(2, "agente");
        assertNotNull(result);
        assertEquals(StatoOfferta.ACCETTATA, result.getStato());
        assertNotNull(result.getDataRisposta());
        assertEquals(StatoNegoziazione.CHIUSA_ACCETTATA, chat.getStatoNegoziazione());
        assertEquals(StatoOfferta.RIFIUTATA, o1.getStato());
        assertEquals(StatoOfferta.RIFIUTATA, o3.getStato());
        verify(offertaRepository, times(1)).findById(2);
        verify(offertaRepository, times(1)).save(result);
    }
\end{lstlisting}    

\newpage 
\textbf{Inserimento di un offerta con valore positivo}
\begin{table}[!h]
    \centering
    \small 
    \begin{tabular}{|p{0.5cm}|p{3cm}|p{5cm}|p{5cm}|}
    \hline
    \textbf{id} & \textbf{Precond} & \textbf{Input} & \textbf{Output aspettato} \\
    \hline
    19 & Nella piattaforma è presente un immobile. L'utente è loggato & importo=550.0, \newline note=Prova
    & response.code=200 \newline response.importoOfferto=550.0 \newline response.note=Prova \\
    \hline
    \end{tabular} 
\end{table}

\begin{lstlisting}[style=react]
    @Test
    @DisplayName("Inserimento di un offerta con valore positivo")
    void makeOffer_positiveAmount_returnsOffertaDTO() {
        // Arrange
        Double importo = 150.0;
        String note = "Prova";
        Offerta offerta = new Offerta(chat, offerente, importo, note);
        offerta.setOffertaId(10);
        when(authentication.getName()).thenReturn(username);
        when(chatService.creaOfferta(chatId, importo, note, username)).thenReturn(offerta);
        ResponseEntity<OffertaDTO> response = chatController.makeOffer(chatId, importo, note, authentication);
        assertEquals(200, response.getStatusCodeValue());
        assertNotNull(response.getBody());
        OffertaDTO dto = response.getBody();
        assertEquals(importo, dto.getImportoOfferto());
        assertEquals(chatId, dto.getChatId());
        assertEquals(offerente.getUsername(), dto.getOfferenteNome());
        verify(chatService, times(1)).creaOfferta(chatId, importo, note, username);
    }
\end{lstlisting}    
 
\newpage 
\textbf{Inserimento di un offerta con valore negativo}
\begin{table}[!h]
    \centering
    \small 
    \begin{tabular}{|p{0.5cm}|p{3cm}|p{5cm}|p{5cm}|}
    \hline
    \textbf{id} & \textbf{Precond} & \textbf{Input} & \textbf{Output aspettato} \\
    \hline
    19 & Nella piattaforma è presente un immobile. L'utente è loggato & importo=-50.0, \newline note=Negativo
    & response.code=400 \\
    \hline
    \end{tabular} 
\end{table}

\begin{lstlisting}[style=react]
    @Test
    @DisplayName("Inserimento di un offerta con valore negativo")
    void makeOffer_negativeAmount_returnsBadRequest() {
        // Arrange
        Double importo = -50.0;
        String note = "Negativo";
        when(authentication.getName()).thenReturn(username);
        when(chatService.creaOfferta(chatId, importo, note, username)).thenReturn(null);
        // Act
        ResponseEntity<OffertaDTO> response = chatController.makeOffer(chatId, importo, note, authentication);
        // Assert
        assertEquals(400, response.getStatusCodeValue());
        assertNull(response.getBody());
        verify(chatService, times(1)).creaOfferta(chatId, importo, note, username);
    }
\end{lstlisting}    
 
\newpage