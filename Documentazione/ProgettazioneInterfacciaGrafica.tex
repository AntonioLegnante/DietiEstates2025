In questa sezione si vuole descrivere come è stata pensata l'interfaccia e perché sono state adottate
determinate scelte progettuali piuttosto che altre.

L'interfaccia di DietiEstates25 è strutturata come un'applicazione web moderna con una barra di 
navigazione principale e un'area centrale dedicata alla gestione e visualizzazione degli immobili.
\\\\
\textbf{Navigazione}

La navigazione è organizzata tramite un menu principale che consente l'accesso diretto alle 
funzionalità principali del sistema. Quando l'utente non è autenticato, il menu è limitato alle 
pagine Home, dalla quale è possibile effettuare la ricerca degli immobili, e alle funzionalità di 
Registrazione e Accesso.

Un utente autenticato dispone di ulteriori voci di menu; in particolare, un agente immobiliare ha 
accesso alla funzionalità di inserimento di un nuovo immobile. Tutti gli utenti autenticati possono 
inoltre accedere allo storico delle chat associate ai singoli immobili tramite un'apposita voce 
del menu.
\\\\
\textbf{Design Visivo e Scelte Cromatiche}

L'interfaccia di DietiEstates25 è stata progettata secondo principi di design minimalista e 
funzionale. Durante la fase iniziale di sviluppo, era stato adottato uno schema cromatico con 
sfondi blu intensi; tuttavia, questa scelta è stata successivamente rivista in favore di uno 
schema più neutro e leggero. L'obiettivo è stato quello di non appesantire visivamente l'interfaccia, 
evitando di affaticare l'utente durante sessioni prolungate di navigazione.

La palette cromatica finale privilegia tonalità chiare e neutre per gli sfondi, con l'utilizzo 
strategico del colore blu brillante esclusivamente per gli elementi interattivi principali, quali 
pulsanti di azione e call-to-action. Questa scelta crea un contrasto efficace che guida 
naturalmente l'attenzione dell'utente verso le funzionalità più rilevanti, migliorando la 
navigabilità complessiva del sistema.

L'implementazione è stata realizzata mediante il framework Tailwind CSS, che ha permesso di 
ottenere transizioni fluide, animazioni delicate e una coerenza stilistica su tutti i componenti 
dell'interfaccia. L'utilizzo di Tailwind ha inoltre facilitato la creazione di un design 
responsivo, garantendo un'esperienza utente ottimale su dispositivi di diverse dimensioni.
\\\\
\textbf{Visualizzazione degli Immobili}

Una delle scelte progettuali più significative riguarda la modalità di presentazione degli 
immobili nei risultati di ricerca. Anziché adottare la soluzione convenzionale di una mappa 
unica contenente tutti gli immobili disponibili rappresentati tramite pin, si è optato per un 
approccio card-based con mappe individuali.

Ogni immobile è presentato tramite una scheda (card) dedicata che include una mappa specifica 
mostrante esclusivamente la posizione di quell'immobile. Questa scelta progettuale risponde a 
diverse esigenze:

\begin{itemize}
    \item \textbf{Riduzione della confusione visiva}: Una mappa unica con decine o centinaia di pin 
    sovrapposti risulta difficile da interpretare, specialmente in aree urbane ad alta densità 
    immobiliare. La visualizzazione individuale elimina completamente questo problema.
    
    \item \textbf{Localizzazione geografica precisa}: La mappa dedicata permette all'utente di 
    comprendere immediatamente il contesto urbano circostante l'immobile, identificando servizi, 
    trasporti e punti di interesse nelle immediate vicinanze.
    
    \item \textbf{Scansionabilità migliorata}: Gli utenti possono scorrere rapidamente i risultati 
    ottenendo un'immediata comprensione sia delle caratteristiche dell'immobile sia della sua 
    collocazione geografica, senza dover continuamente interagire con zoom e pan su una mappa 
    condivisa.
    
    \item \textbf{Focus contestuale}: Ogni card fornisce un contesto completo e autonomo, riducendo 
    il carico cognitivo necessario per correlare informazioni distribuite su interfacce diverse.
\end{itemize}

Le card sono state progettate con un layout pulito e gerarchico, presentando le informazioni più 
rilevanti (prezzo, tipologia, indirizzo) in posizione prominente, mentre dettagli secondari sono 
accessibili espandendo la card o navigando verso la pagina dedicata dell'immobile.
\\\\
\textbf{Sistema di Gestione delle Offerte}

Un elemento centrale dell'applicazione è il sistema di gestione delle offerte immobiliari, 
progettato per facilitare la comunicazione e la negoziazione tra utenti e agenti immobiliari. 
Il design di questa funzionalità è stato sviluppato con particolare attenzione all'usabilità e 
alla chiarezza dello stato della trattativa.

Quando un utente invia un'offerta per un immobile, questa viene visualizzata all'interno di una 
card dedicata che mostra l'importo proposto, la data e l'ora di invio, e un badge di stato. 
L'interfaccia utente presenta tre pulsanti di azione chiari e distinti: "Accetta", "Controfferta" 
e "Rifiuta", ciascuno accompagnato da icone intuitive che ne rafforzano il significato.

Dal punto di vista dell'agente immobiliare, l'interfaccia è stata progettata per fornire una 
panoramica immediata di tutte le negoziazioni attive. La sezione "Le tue negoziazioni" presenta 
una lista delle trattative in corso, dove ogni elemento mostra informazioni essenziali quali il 
nome dell'utente interessato, l'immobile di riferimento, l'ultima offerta ricevuta e lo stato 
attuale della negoziazione.

\textbf{Codifica Cromatica degli Stati}

Una delle decisioni progettuali più efficaci riguarda l'utilizzo di un sistema di codifica 
cromatica per comunicare istantaneamente lo stato di un'offerta. Questo approccio sfrutta i 
principi della psicologia del colore e delle convenzioni universalmente riconosciute:

\begin{itemize}
    \item \textbf{Verde (Accettata)}: Quando un'offerta viene accettata, la card assume una 
    colorazione verde con bordo verde, comunicando immediatamente un risultato positivo. Il verde 
    è universalmente associato a conferma, successo e procedere, rendendo il feedback immediato e 
    inequivocabile.
    
    \item \textbf{Blu (Controfferta)}: Se l'agente decide di formulare una controfferta, la card 
    mantiene la colorazione blu standard, coerente con il tema cromatico dell'applicazione. Il blu 
    comunica una condizione neutra, indicando che la negoziazione è ancora in corso e richiede 
    ulteriore interazione.
    
    \item \textbf{Rosso (Rifiutata)}: In caso di rifiuto, la card assume una tonalità rossa con 
    bordo rosso, segnalando chiaramente la conclusione negativa della trattativa. Il rosso è 
    universalmente riconosciuto come indicatore di stop, errore o stato negativo.
    
\end{itemize}

Questo sistema di feedback visivo immediato riduce drasticamente il tempo necessario per 
comprendere lo stato di una negoziazione, permettendo agli utenti di elaborare le informazioni 
a colpo d'occhio senza dover leggere descrizioni testuali dettagliate. La codifica cromatica si 
applica sia al bordo della card che al badge di stato, creando una ridondanza informativa che 
rinforza il messaggio e lo rende accessibile anche in condizioni di visualizzazione non ottimali.

\textbf{Flusso di Interazione delle Offerte}

Il flusso di interazione è stato progettato per essere lineare e privo di ambiguità. Quando un 
utente visualizza le proprie offerte, può immediatamente comprenderne lo stato grazie alla 
codifica cromatica. Se l'offerta è ancora in attesa, l'utente può scegliere di attendere una 
risposta o eventualmente modificarla. Se è stata fatta una controfferta, l'utente visualizza 
il nuovo importo proposto dall'agente e può decidere se accettarlo, rifiutarlo o formulare a 
sua volta una nuova proposta.

Dal lato dell'agente, l'interfaccia fornisce un dashboard centralizzato dove tutte le negoziazioni 
attive sono visibili contemporaneamente. Ogni card di negoziazione presenta un gradiente decorativo 
che aggiunge profondità visiva senza compromettere la leggibilità delle informazioni. Un badge 
"In corso" di colore blu chiaro indica le trattative ancora aperte, mentre un contatore mostra 
il numero totale di offerte ricevute per quell'immobile specifico.

La gestione delle controfferte è stata implementata con particolare attenzione alla trasparenza: 
quando l'agente formula una controfferta, l'utente riceve una notifica e può visualizzare sia 
l'offerta originale che quella proposta dall'agente, facilitando il confronto e la decisione 
informata.

\textbf{Elementi di Micro-Interazione}

Particolare cura è stata dedicata alle micro-interazioni che accompagnano le azioni dell'utente. 
Quando si clicca su uno dei pulsanti di azione (Accetta, Controfferta, Rifiuta), l'interfaccia 
fornisce feedback immediato attraverso transizioni smooth implementate con Tailwind CSS. La card 
cambia colore gradualmente, comunicando che l'azione è stata registrata con successo. Questo tipo 
di feedback rassicura l'utente e riduce l'incertezza tipica delle interfacce che non forniscono 
conferme visive immediate.

Le icone associate ai pulsanti (spunta per accettazione, frecce contrapposte per controfferta, 
cerchio barrato per rifiuto) sono state selezionate seguendo convenzioni consolidate nell'ambito 
dell'interaction design, garantendo riconoscibilità immediata anche per utenti meno esperti.
\\\\
\textbf{Motivazioni delle Scelte e Riduzione dell'Attrito Cognitivo}

L'interfaccia privilegia una disposizione semplice e lineare al fine di ridurre il carico cognitivo 
degli utenti, che non sono necessariamente esperti informatici. Il design è stato pensato seguendo 
principi di progressive disclosure: le informazioni vengono rivelate gradualmente in base alle 
necessità dell'utente, evitando di sovraccaricare lo schermo con opzioni e dati non immediatamente 
rilevanti.

La pagina principale presenta una barra di ricerca con filtri essenziali e immediati, quali città, 
tipologia dell'operazione (affitto o vendita), prezzo minimo e prezzo massimo. Questa selezione 
minima di filtri rappresenta i criteri di ricerca più comuni secondo le statistiche di utilizzo 
delle piattaforme immobiliari. Ricerche più specifiche e dettagliate possono essere effettuate 
tramite l'icona "Altri filtri", che apre un pannello modale contenente opzioni avanzate quali 
numero di stanze, metratura, piano, classe energetica e numero di bagni.

L'utilizzo di transizioni smooth e micro-interazioni implementate tramite Tailwind CSS contribuisce 
a creare un'esperienza fluida e naturale. Elementi quali l'hover sulle card, l'apertura dei menu 
e la visualizzazione dei filtri avanzati sono accompagnati da animazioni delicate che guidano 
l'attenzione senza distrarre, riducendo l'attrito cognitivo durante la navigazione.

L'utilizzo di maschere guidate e campi obbligatori riduce la possibilità di inserimento di dati non 
validi, come date incoerenti o importi errati, migliorando l'affidabilità complessiva del sistema. 
I messaggi di validazione sono posizionati in prossimità dei campi interessati e utilizzano un 
linguaggio chiaro e orientato alla soluzione.

La pagina principale, fino all'esecuzione di una ricerca, rimane intenzionalmente vuota per limitare 
la quantità di informazioni visualizzate e prevenire il disorientamento dei nuovi utenti. Questa 
scelta, nota come "zero state design", permette di mantenere il focus sulla funzionalità di ricerca 
e riduce l'ansia decisionale che potrebbe derivare dalla presentazione immediata di numerose opzioni.

L'architettura informativa è stata progettata seguendo il principio del "mobile-first design", 
garantendo che l'esperienza sia ottimale anche su dispositivi con schermi ridotti. La 
responsività non è stata implementata come semplice adattamento, ma come una riprogettazione 
consapevole delle interazioni per ogni breakpoint, assicurando usabilità su qualsiasi dispositivo.

In sintesi, ogni scelta progettuale è stata guidata dall'obiettivo di creare un'interfaccia che 
sia al contempo potente nelle funzionalità e accessibile nell'utilizzo, riducendo sistematicamente 
ogni forma di attrito cognitivo e massimizzando la chiarezza comunicativa attraverso l'uso 
strategico di colore, tipografia, spacing e animazioni.