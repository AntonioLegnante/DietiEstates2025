\textbf{Architettura complessiva}
L’applicazione adotta un’architettura client–server organizzata secondo il modello a 3 livelli (3-tier), strutturata come Single Page Application (SPA) e progettata seguendo il pattern MVC.
Al primo accesso il server invia al client l’intera applicazione Web (HTML, CSS, JavaScript).
Da quel momento l’applicazione viene eseguita interamente nel browser, e il server entra in gioco soltanto per fornire o aggiornare i dati tramite API RESTful, che restituiscono JSON.
Il rendering dell’interfaccia utente avviene lato client (Client-Side Rendering).
\\ \\
\textbf{Architettura del Server (Backend)}
Il lato server adotta un’architettura ispirata al pattern MVC, ma limitata alle componenti Model 
e Controller, poiché la View è delegata al client.
\\ \\
\textbf{Model}
Il model comprende:
\begin{itemize}
    \item Entità: rappresentano le strutture dati persistenti (immobili, utenti, chat, ecc...)
    \item Logica di business (servizi): implementano le regole applicative 
    (inserimento di un immobile, autenticazione, ricerca immobili, apertura chat…).
\end{itemize}
Le entità costituiscono il nucleo del livello dati.
I servizi fungono da ponte tra i Controller e il database, incapsulando la logica
applicativa.
\\ \\
\textbf{Controller}
Il Controller:
\begin{itemize}
    \item riceve le richieste HTTP dai client (routing),
    \item le interpreta 
    \item invoca la logica dei servizi
    \item restituisce risposte JSON
\end{itemize}
Non produce HTML (la View è lato client).
Il backend espone esclusivamente servizi RESTful.
\\ \\
\textbf{Sistema di persistenza dei dati}
Il sistema di persistenza dei dati è costituito:
\begin{itemize}
    \item Sistema di persistenza dei dati testuali: gestito tramite un ORM 
    (Object–Relational Mapping), che assicura una netta separazione tra la 
    logica applicativa e il database relazionale. L’ORM consente di modellare le 
    entità come oggetti, gestire le relazioni in modo trasparente e semplificare 
    le operazioni CRUD.
    \item Sistema di persistenza delle immagini gestito tramite un apposito Database.
    In questo modo si garantiscono migliori prestazioni, minor carico sul database 
    e una maggiore leggerezza complessiva del sistema. Nel database viene 
    mantenuto solo il riferimento (path o URL) all’immagine.
\end{itemize}
\textbf{Sistema di autenticazione e autorizzazione}
L'applicazione adotta un meccanismo stateless di autenticazione e autorizzazione 
basato su JWT (JSON Web Token). Al momento del login, il server genera un token 
firmato contenente le informazioni essenziali sull’utente, come identificatore, 
ruolo e tempo di scadenza. Il client memorizza il token localmente e lo include in 
tutte le successive richieste HTTP verso le API RESTful, tramite l’header 
Authorization. Il server verifica la validità e l’integrità del token ad ogni 
richiesta, senza mantenere sessioni lato backend. Questo approccio garantisce:
sicurezza, scalabilità e flessibilità.
\\ \\
\textbf{Architettura del client}
Il client costituisce la View dell’applicazione e gestisce:
\begin{itemize}
    \item il rendering dell’interfaccia (Client-Side Rendering),
    \item la gestione dello stato,
    \item le interazioni dell’utente,
    \item le chiamate alle API REST del backend.
\end{itemize}
\textbf{Sintesi dei 3 livelli (3-tier)}
\begin{enumerate}
    \item Presentation Tier (Client / Browser)
        \begin{itemize}
            \item View
            \item SPA renderizzata lato client
            \item Gestione dell’interfaccia e delle interazioni
            \item Memorizzazione e utilizzo dei token JWT
        \end{itemize}
    \item Application Tier (Server / Logica di business)
        \begin{itemize}
            \item Controller
            \item Servizi (Business Logic)
            \item Gestione sicurezza, verifica dei token JWT ad ogni richiesta
        \end{itemize}
    \item Data Tier (Database)
        \begin{itemize}
            \item Dati testuali
            \item Dati multimediali (immagini)
            \item Operazioni CRUD sui dati
        \end{itemize}
\end{enumerate}

\textbf{Ciclo delle richieste}
\begin{enumerate}
    \item L'utente invia una richiesta HTTP/HTTPS (GET o POST)
    \item DispatcherServlet intercetta la richiesta e la inoltra al controller 
    appropriato. (Front Controller design pattern)
    \item il controller elabora la logica, interagisce con il model e prepara i dati
    \item il controller restituisce i dati in formato JSON
\end{enumerate}
