\newpage
\textbf{Test Accettare la seconda offerta su tre}
\begin{table}[!h]
    \centering
    \small 
    \begin{tabular}{|p{0.5cm}|p{3cm}|p{5cm}|p{5cm}|}
    \hline
    \textbf{id} & \textbf{Precond} & \textbf{Input} & \textbf{Output aspettato} \\
    \hline
    17 & Nella piattaforma è presente un immobile. Su questa chat sono state caricate 3 offerte & idOfferta=2 \newline username=agente & 
    result!=null, \newline result.dataRisposta!=null, \newline offerta1.StatoOfferta=CHIUSA\_AC -CETTATA, \newline offerta3.StatoOfferta=CHIUSA\_RI -FIUTATA, 
    \newline offerta3.StatoOfferta=CHIUSA\_RI -FIUTATA \\
    \hline
    \end{tabular} 
\end{table}

\begin{lstlisting}[style=react]
    @Test
    @DisplayName("Accettare la seconda offerta su tre")
    void accettaOffertaSeconda() {
        //Arrange
        when(offertaRepository.findById(2)).thenReturn(Optional.of(o2));
        when(utenteRepository.findByUsername("agente")).thenReturn(Optional.of(vendor));
        when(offertaRepository.save(any(Offerta.class))).thenAnswer(inv -> inv.getArgument(0));
        //Act
        Offerta result = chatService.accettaOfferta(2, "agente");
        assertNotNull(result);
        assertEquals(StatoOfferta.ACCETTATA, result.getStato());
        assertNotNull(result.getDataRisposta());
        assertEquals(StatoNegoziazione.CHIUSA_ACCETTATA, chat.getStatoNegoziazione());
        assertEquals(StatoOfferta.RIFIUTATA, o1.getStato());
        assertEquals(StatoOfferta.RIFIUTATA, o3.getStato());
        verify(offertaRepository, times(1)).findById(2);
        verify(offertaRepository, times(1)).save(result);
    }
\end{lstlisting}    

\newpage
\textbf{Test Accettare la terza offerta su tre}
\begin{table}[!h]
    \centering
    \small 
    \begin{tabular}{|p{0.5cm}|p{3cm}|p{5cm}|p{5cm}|}
    \hline
    \textbf{id} & \textbf{Precond} & \textbf{Input} & \textbf{Output aspettato} \\
    \hline
    19 & Nella piattaforma è presente un immobile. Su questa chat sono state caricate 3 offerte & idOfferta=3 \newline username=agente & 
    result!=null, \newline result.dataRisposta!=null, \newline offerta1.StatoOfferta=CHIUSA\_RI -FIUTATA, \newline offerta3.StatoOfferta=CHIUSA\_AC -CETTATA, 
    \newline offerta2.StatoOfferta=CHIUSA\_RI -FIUTATA \\
    \hline
    \end{tabular} 
\end{table}

\begin{lstlisting}[style=react]
    @Test
    @DisplayName("Accettare la terza offerta su tre")
    void accettaOffertaSeconda() {
        //Arrange
        when(offertaRepository.findById(2)).thenReturn(Optional.of(o2));
        when(utenteRepository.findByUsername("agente")).thenReturn(Optional.of(vendor));
        when(offertaRepository.save(any(Offerta.class))).thenAnswer(inv -> inv.getArgument(0));
        //Act
        Offerta result = chatService.accettaOfferta(2, "agente");
        assertNotNull(result);
        assertEquals(StatoOfferta.ACCETTATA, result.getStato());
        assertNotNull(result.getDataRisposta());
        assertEquals(StatoNegoziazione.CHIUSA_ACCETTATA, chat.getStatoNegoziazione());
        assertEquals(StatoOfferta.RIFIUTATA, o1.getStato());
        assertEquals(StatoOfferta.RIFIUTATA, o3.getStato());
        verify(offertaRepository, times(1)).findById(2);
        verify(offertaRepository, times(1)).save(result);
    }
\end{lstlisting}    

