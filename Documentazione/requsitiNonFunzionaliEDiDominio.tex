I requisiti non funzionali descrivono vincoli di qualità applicabili all'intero sistema.
I requisiti di dominio derivano dal contesto reale in cui il sistema opera:
leggi, regolamenti, abitudini, regole del settore.
Sono di seguito elencati i requisiti non funzionali e di dominio

\textbf{Usabilità}

\begin{itemize}
    \item Un utente non esperto deve poter registrare un immobile senza consultare documentazione esterna.
    \item L’interfaccia deve essere accessibile tramite browser web moderni (Chrome, Firefox).
\end{itemize} 

\textbf{Sicurezza}

\begin{itemize}
    \item L’accesso al sistema deve avvenire tramite autenticazione con credenziali.
    \item Le password devono essere memorizzate in forma cifrata (hash).
    \item Un utente può accedere solo ai dati per cui è autorizzato
\end{itemize}

\textbf{Prestazioni}

\begin{itemize}
    \item Il sistema deve rispondere alle operazioni principali (login, visualizzazione immobili, registrazione contratto) entro 2 secondi nel 95\% dei casi.
    \item Il sistema deve supportare almeno 100 utenti contemporanei senza degrado significativo delle prestazioni.
\end{itemize}

\textbf{Manutenibilità}
\begin{itemize}
    \item Il sistema deve essere progettato in modo modulare per facilitare l’aggiunta di nuove funzionalità.
    \item Le componenti devono essere debolmente accoppiate per consentire modifiche locali senza impatti sull’intero sistema.
    \item Il codice deve essere organizzato secondo un’architettura che favorisca la manutenibilità.
\end{itemize}

\textbf{Aspetti economici}

\begin{itemize}
    \item Il canone di locazione/ prezzo di vendità deve essere espresso in euro.
\end{itemize}

\textbf{Gestione immobili}

\begin{itemize}
    \item Ogni immobile deve essere identificato in modo univoco all’interno del sistema.
\end{itemize}

Aggiungiamo una nota anche legata ai vincoli e assunzioni di progetto; Specifichiamo che questi 
non sono requisiti non funzionali puri

\textbf{Vincoli di progetto}
\begin{itemize}
    \item Il sistema deve essere sviluppato utilizzando tecnologie note al team.
    \item L’architettura deve rimanere semplice per facilitare manutenzione e sviluppo.
\end{itemize}



