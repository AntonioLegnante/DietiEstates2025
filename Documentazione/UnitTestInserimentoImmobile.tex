\textbf{Test Creazione Immobile valido}
\begin{table}[!h]
    \centering
    \small 
    \begin{tabular}{|p{0.5cm}|p{3cm}|p{5cm}|p{5cm}|}
    \hline
    \textbf{id} & \textbf{Precond} & \textbf{Input} & \textbf{Output aspettato} \\
    \hline
    7 & Agente immobiliare loggato correttamente & titolo=Appartamento Lusso, \newline descrizione=Bellissimo appartamento, \newline prezzo=350000.0, \newline dimensione=120 mq,
    \newline citta=Napoli \newline indirizzo=Via Roma 123 \newline affitto=false, \newline vendita=true, \newline numeroStanze=3, \newline piano=3,
    \newline classeEnergetica=B \newline garage=true \newline numeroBagni=2 \newline galleryImages=imageFile \newline galleryImages=galleryImages & 
    result!=null \newline result.titolo=Appartamento Lusso \newline result.prezzo=350000.0 \newline result.citta=Napoli \\
    \hline
    \end{tabular} 
\end{table}


\begin{lstlisting}[style=react]
    @Test
    @DisplayName("Test Creazione Immobile valido")
    void testCreateImmobileSuccess() throws Exception {
        // Arrange
        agente = new Utente();
        agente.setId(1);
        agente.setUsername("agent1");
        agente.setRuolo("agente immobiliare");
        when(utenteRepository.findByUsername("agent1")).thenReturn(Optional.of(agente));
        when(minioService.uploadFile(imageFile))
                .thenReturn("https://minio.example.com/img1.jpg");
        when(minioService.uploadMultipleFiles(galleryImages))
                .thenReturn(Arrays.asList("https://minio.example.com/gallery1.jpg"));
        when(immobileRepository.save(any(Immobile.class))).thenReturn(immobile);
        // Act
        Immobile result = immobileService.createImmobile(
                "Appartamento Lusso", "Bellissimo appartamento", 350000.0,
                "120 mq", "Napoli", "Via Roma 123",
                false, true, 3, "3", "B", true, 2,
                imageFile, galleryImages, "agent1"
        );
        // Assert
        assertNotNull(result);
        assertEquals("Appartamento Lusso", result.getTitolo());
        assertEquals(350000.0, result.getPrezzo());
        assertEquals("Napoli", result.getCitta());
        assertEquals("agent1", result.getUtente().getUsername());
        verify(immobileRepository, times(1)).save(any(Immobile.class));
    }
\end{lstlisting}
\newpage 

\textbf{Test Creazione Immobile senza effettuare l'accesso - Test del back end}
\begin{table}[!h]
    \centering
    \small 
    \begin{tabular}{|p{0.5cm}|p{3cm}|p{5cm}|p{5cm}|}
    \hline
    \textbf{id} & \textbf{Precond} & \textbf{Input} & \textbf{Output aspettato} \\
    \hline
    8 & Agente non loggato; test del metodo back end & titolo=Appartamento, \newline descrizione=Descrizione, \newline prezzo=100000.0, \newline dimensione=100 mq,
    \newline citta=Napoli \newline indirizzo=Via Test \newline affitto=false, \newline vendita=true, \newline numeroStanze=2, \newline piano=1,
    \newline classeEnergetica=A \newline garage=false \newline numeroBagni=1 \newline galleryImages=imageFile \newline galleryImages=galleryImages & 
    Lancia Exception \\
    \hline
    \end{tabular} 
\end{table}


\begin{lstlisting}[style=react]
    @Test
    @DisplayName("Test Creazione Immobile con agente non trovato")
    void testCreateImmobileFailureAgentNotFound() throws Exception {
        // Arrange
        when(utenteRepository.findByUsername("nonexistent"))
                .thenReturn(Optional.empty());
        when(minioService.uploadFile(imageFile))
                .thenReturn("https://minio.example.com/img1.jpg");

        // Act & Assert
        assertThrows(Exception.class, () -> {
            immobileService.createImmobile(
                    "Appartamento", "Descrizione", 100000.0,
                    "100 mq", "Napoli", "Via Test",
                    false, true, 2, "1", "A", false, 1,
                    imageFile, galleryImages, "nonexistent"
            );
        });
    }
\end{lstlisting}

\newpage

\textbf{Test Creazione Immobile valido senza gallery}
\begin{table}[!h]
    \centering
    \small 
    \begin{tabular}{|p{0.5cm}|p{3cm}|p{5cm}|p{5cm}|}
    \hline
    \textbf{id} & \textbf{Precond} & \textbf{Input} & \textbf{Output aspettato} \\
    \hline
    9 & Agente immobiliare loggato correttamente & titolo=Appartamento Lusso, \newline descrizione=Bellissimo appartamento, \newline prezzo=350000.0, \newline dimensione=120 mq,
    \newline citta=Napoli \newline indirizzo=Via Roma 123 \newline affitto=false, \newline vendita=true, \newline numeroStanze=3, \newline piano=3,
    \newline classeEnergetica=B \newline garage=true \newline numeroBagni=2 \newline galleryImages=imageFile \newline galleryImages=null & 
    result!=null \newline result.titolo=Appartamento Lusso \\
    \hline
    \end{tabular} 
\end{table}


\begin{lstlisting}[style=react]
   @Test
    @DisplayName("Test Creazione Immobile senza gallery")
    void testCreateImmobileWithoutGallery() throws Exception {
        // Arrange
        when(utenteRepository.findByUsername("agent1")).thenReturn(Optional.of(agente));
        when(minioService.uploadFile(imageFile))
                .thenReturn("https://minio.example.com/img1.jpg");
        when(immobileRepository.save(any(Immobile.class))).thenReturn(immobile);

        // Act
        Immobile result = immobileService.createImmobile(
                "Appartamento Lusso", "Bellissimo appartamento", 350000.0,
                "120 mq", "Napoli", "Via Roma 123",
                false, true, 3, "3", "B", true, 2,
                imageFile, null, "agent1"
        );

        // Assert
        assertNotNull(result);
        assertEquals("Appartamento Lusso", result.getTitolo());
    }
\end{lstlisting}

\newpage

\textbf{Test Creazione Immobile valido senza gallery}
\begin{table}[!h]
    \centering
    \small 
    \begin{tabular}{|p{0.5cm}|p{3cm}|p{5cm}|p{5cm}|}
    \hline
    \textbf{id} & \textbf{Precond} & \textbf{Input} & \textbf{Output aspettato} \\
    \hline
    10 & Agente immobiliare loggato correttamente & titolo=Appartamento Lusso, \newline descrizione=Bellissimo appartamento, \newline prezzo=350000.0, \newline dimensione=120 mq,
    \newline citta=Napoli \newline indirizzo=Via Roma 123 \newline affitto=false, \newline vendita=true, \newline numeroStanze=3, \newline piano=3,
    \newline classeEnergetica=B \newline garage=true \newline numeroBagni=2 \newline galleryImages=null \newline galleryImages=galleryImages & 
    lancia Exception \\
    \hline
    \end{tabular} 
\end{table}


\begin{lstlisting}[style=react]
    @Test
    @DisplayName("Test Creazione Immobile senza coverImage")
    void testCreateImmobileWithoutCoverImage() throws Exception {
        // Arrange
        when(utenteRepository.findByUsername("agent1")).thenReturn(Optional.of(agente));
        when(minioService.uploadFile(null))
                 .thenThrow(new Exception("Cover image is required"));
        when(minioService.uploadMultipleFiles(galleryImages))
                .thenReturn(Arrays.asList("https://minio.example.com/gallery1.jpg"));
        when(immobileRepository.save(any(Immobile.class))).thenReturn(immobile);

        // Act
        assertThrows(Exception.class, () -> {
            immobileService.createImmobile(
                    "Appartamento Lusso", "Bellissimo appartamento", 350000.0,
                    "120 mq", "Napoli", "Via Roma 123",
                    false, true, 3, "3", "B", true, 2,
                    null, galleryImages, "agent1"
            );
        });
    }
\end{lstlisting}
\newpage
 